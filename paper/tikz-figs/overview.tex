% Base dimensions (essentially, everything is derived from WIDTH)
\setlength{\WIDTH}{\linewidth}
\setlength{\HEIGHT}{0.5\WIDTH}

% Visual aesthetic dimensions
\newcommand{\innerSep}{4pt}
\newcommand{\lineDist}{2pt}

\tikzset{subCirc/.style={fill=bggray, draw=gray!20}}
\tikzset{dataStyle/.style={draw, semithick, dashed}}
\tikzset{varStyle/.style={draw, thin, solid, double}}
\tikzset{infoStyle/.style={draw, lfd3, very thick, -Stealth}}
\tikzset{stepArrow/.style={line width=2mm, gray!50,
                           -{Triangle Cap []. Fast Triangle[] Fast Triangle[]}}}

\def\circShift{1.5em}
\def\varCtrlDist{0.5cm}
\def\joNodeDist{1.0cm}
\def\vcont{\raisebox{0.5em}{\(\vdots\)}}


\begin{tikzpicture}[remember picture]
  %\draw[help lines, lightgray] (0,0) grid (\WIDTH, \HEIGHT);
  \coordinate (tl) at (0, \HEIGHT);
  \coordinate (tr) at (\WIDTH, \HEIGHT);
  \coordinate (bl) at (0, 0);
  \coordinate (br) at (\WIDTH, 0);

%  \fill[red] (tl) circle (2pt);
%  \fill[red] (br) circle (2pt);

  \draw (0,\HEIGHT) node (qbox) [inner sep=0pt, rotate=90, anchor=north east] {
    \begin{minipage}{0.8\HEIGHT}
      \begin{querybox}{Query}
        \begin{lstlisting}[style=query]
SELECT * FROM A as a1, A as a2, D
WHERE a1.a=D.a AND a2.b=D.b
               AND D.c > 5;
        \end{lstlisting}
      \end{querybox}
    \end{minipage} 
  };

\coordinate (tstart) at (qbox.south east);
\calcLength(tstart,tr){tspace}
\newlength{\dist}\setlength{\dist}{\tspace pt}
\newlength{\dbwidth}
% -2mm: WTFF???? -> Must be some innersep or something, but I have no 
% idea where I could have missed that
\pgfmathsetmacro{\tmp}{(\dist-(\facc\linewidth+\wd\stepa+\wd\stepb))/3-2mm}
\setlength{\dist}{\tmp pt}
\pgfmathsetmacro{\tmp}{\dist+0.5*(\faca\linewidth+\facb\linewidth)}
\setlength{\dbwidth}{\tmp pt}

  \node[anchor=north west, fill=bggray, minimum width=\dbwidth] (db)
        at ($(tstart)+(\dist, 0)$) { 
        \begin{tabular}{ll}
          Tables &\(\{A, B, C, D\}\)\\
           Aliases &\(\{a_{1}, a_{2}, B, C, D\}\)\\
        Attributes &\(a_{1}.a, a_{1}.b,\ldots, D.c, D.d\)
        \end{tabular}
        };
  
  \node[very thin, anchor=north west, fill=bggray, inner sep=0pt] (data1) at 
       (db.south west) { \usebox{\stepa} };
%
  \node[very thin, anchor=north east, fill=bggray, inner sep=0pt] (data2) at
       (db.south east) { \usebox{\stepb} };
%
   % Zeilenweise Linearisierung
  % Join Graph, Tree Structure, Table Indices, Table Selectivities
  \newcommand{\ca}[1]{\mathcolor{lfd1}{#1}}
  \newcommand{\cb}[1]{\mathcolor{lfd2}{#1}}
  \newcommand{\cc}[1]{\mathcolor{lfd4}{#1}}
  \node[anchor=north west, inner sep=0pt] (data3) at 
                                ($(db.north east) + (\dist, 0)$) { 
       \begin{minipage}{\facc\WIDTH}
         \begin{displaymath}
           \footnotesize\begin{bNiceMatrix}
           \ca0\\\ca0\\\ca3\\\ca{1.0}\\\ca{1.0}\\\ca{0.75}\\\ca1\\\ca0\\\cb0\\\cb0\\\cb1\\\cb0\\\cb0\\\cb0\\\cb1\\\cb0\\\cc0\\\cc1\\\cc0\\\cc0\\\cc1\\\cc1\\\cc1\\\cc0
           \end{bNiceMatrix}
         \end{displaymath}
       \end{minipage}
  };

  \coordinate (m1) at ($(data3.south east)!0.85!(data3.north east)$);
  \coordinate (m2) at ($(data3.south east)!0.50!(data3.north east)$);
  \coordinate (m3) at ($(data3.south east)!0.15!(data3.north east)$);

  %%%%%%%%%%%%%%%%%%%%%%%%% Arrows between elements %%%%%%%%%%%%%%%%%
    \path[infoStyle] ([xshift=\lineDist]qbox.south) -- +(\dist-\lineDist, 0) 
                         node[above, midway]{ Baseline } 
                         node[below, midway]{ Encoding };

  \coordinate (sarrow) at ($(data1.north east)!(qbox.south)!(data1.south east)$);      
  \path[infoStyle] ([xshift=\lineDist]sarrow) -- +(\dist-\lineDist, 0) 
                         node[above, midway]{ Feature } 
                         node[below, midway]{ Reduction };                         

  \coordinate (sarrow) at ($(data2.north east)!(qbox.south)!(data2.south east)$);      
  \path[infoStyle] ([xshift=\lineDist]sarrow) -- +(\dist-\lineDist, 0) 
                         node[above, midway]{ Flattening \& } 
                         node[below, midway]{ Concatenation };                         
  

\yquantdefinebox{dots}[inner sep=0pt]{$\dots$}
\hspace*{\circShift}\begin{yquant}
      qubit {\(\ket{0}\)} q;
      qubit {\raisebox{0.5em}{\(\vdots\)}\hspace{0.5em}} sep;
      qubit {\(\ket{0}\)} p;
 
      [this subcircuit box style={subCirc}, name=sc1]
      subcircuit {
        qubit {} q;
        qubit {} sep;
        qubit {} p;

        [style={draw=lfd1},name=enc1] box {
                    $\mathcolor{lfd1}{\hat{U}_{\text{enc}}}$} (q, sep, p) ;
        [style={draw=lfd1},name=var1] box {
                    $\mathcolor{lfd1}{\hat{U}_{\text{var}}}$} (q, sep, p) ;
        [style={draw=lfd2},name=enc2] box {
                    $\mathcolor{lfd2}{\hat{U}_{\text{enc}}}$} (q, sep, p) ;
        [style={draw=lfd2},name=var2] box {
                    $\mathcolor{lfd2}{\hat{U}_{\text{var}}}$} (q, sep, p) ;
        [style={draw=lfd4},name=enc3] box {
                    $\mathcolor{lfd4}{\hat{U}_{\text{enc}}}$} (q, sep, p) ;
        [style={draw=lfd4},name=var3] box {
                    $\mathcolor{lfd4}{\hat{U}_{\text{var}}}$} (q, sep, p) ;
      } (q, sep, p);
      
%      [ultra thin] barrier (-);
      [value=7mm] hspace q;
      [value=7mm] hspace sep;
      [value=7mm] hspace p;

      [this subcircuit box style={subCirc}, name=sc2]
      subcircuit {
        qubit {} q;
        qubit {} sep;
        qubit {} p;

        [style={draw=lfd1},name=enc1] box {} (q, sep, p) ;
        [style={draw=lfd1},name=var1] box {} (q, sep, p) ;
        [style={draw=lfd2},name=enc2] box {} (q, sep, p) ;
        [style={draw=lfd2},name=var2] box {} (q, sep, p) ;
        [style={draw=lfd4},name=enc3] box {} (q, sep, p) ;
        [style={draw=lfd4},name=var3] box {} (q, sep, p) ;
      } (q, sep, p);

      text { \textbf{\(\cdots\)} } (q, sep, p);
    
      [this subcircuit box style={subCirc}, name=sc3]
      subcircuit {
        qubit {} q;
        qubit {} sep;
        qubit {} p;

        [style={draw=lfd1},name=enc1] box {} (q, sep, p) ;
        [style={draw=lfd1},name=var1] box {} (q, sep, p) ;
        [style={draw=lfd2},name=enc2] box {} (q, sep, p) ;
        [style={draw=lfd2},name=var2] box {} (q, sep, p) ;
        [style={draw=lfd4},name=enc3] box {} (q, sep, p) ;
        [style={draw=lfd4},name=var3] box {} (q, sep, p) ;
      } (q, sep, p);

      [name=meas1] measure q;
      text { \vcont } sep;
      [name=meas4] measure p;
      discard q; discard sep; discard p;

      [value=0.25cm] hspace q;
      [value=0.25cm] hspace sep;
      [value=0.25cm] hspace p;
      
      [style={draw=black},name=pp] box { \rotatebox{90}{Post-Processing}} (q, sep, p);
      text { \vcont } sep;
\end{yquant}

\end{tikzpicture}


\begin{tikzpicture}[remember picture, overlay]
%%%%%%%%%%%%%% Aditional TikZ elements for the circuit %%%%%%%%%%%%%%

%%%%%%%%% Feed variational parameters from measurement into circuit %%%%%%%%%
\coordinate (start_var) at ($(sc1-var1.south) - (0,\varCtrlDist)$);
\coordinate (tmp) at ($(pp.north east)!(meas1.east)!(pp.north east)$);
\coordinate (tr_var) at ($(tmp) + (\varCtrlDist, 0)$);
\coordinate (corner_var) at (start_var -| tr_var);
%\fill[blue] (corner_var) circle (2pt);
%\fill[blue] (tr_var) circle (2pt);
\path[varStyle] (start_var) -| (tr_var);

\foreach \n in {1,4} { 
    \path[varStyle] (meas\n.east) -- ($(pp.north west)!(meas\n.east)!(pp.south west)$)
                    ($(pp.north east)!(meas\n.east)!(pp.north east)$) -- 
                   +(\varCtrlDist, 0);
}

\foreach \n in {1,2,3} { 
  \foreach \m in {1,2,3} { 
    \path[varStyle, -Stealth] ($(sc\n-var\m.south) - (0, \varCtrlDist)$) -- 
                              (sc\n-var\m.south);
  }
}

%%%%%%%%%%% Encode input parameters into circuit %%%%%%%%%%%%%%%
\coordinate (l1) at ($(m1) + (\varCtrlDist, 0)$);
\coordinate (l2) at ($(m3) + (\varCtrlDist, 0)$);
\coordinate (prj) at ($(sc1-enc1.north) + (0, \varCtrlDist)$);
\coordinate (corner_enc) at ($(corner_var)!(prj)!(tr_var)$);
\coordinate (start_enc) at (m1 -| corner_enc);

%\fill[red] (start_enc) circle (2pt);

\foreach \n in {1,2,3} { 
    \path[thick, dataStyle] ([xshift=-\circShift]m\n) -- 
                           ($(start_enc)!(m\n)!(corner_enc)$);
}
\path[dataStyle] (start_enc) -- (corner_enc);

\foreach \n in {1,2,3} { 
  \foreach \m in {1,2,3} { 
    \path[dataStyle, -Stealth] (corner_enc) -| (sc\m-enc\n.north);
  }
}


\node[anchor=south east] at (corner_var.north west) { \textbf{\(\circlearrowright\)} };
%%%%%%%%%%%%%%%%%%%%%%%
\end{tikzpicture}
